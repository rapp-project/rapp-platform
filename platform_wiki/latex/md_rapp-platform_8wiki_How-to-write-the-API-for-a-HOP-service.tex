Currently, we support and maintain R\-A\-P\-P-\/\-A\-P\-I(s) for the following programming languages\-:


\begin{DoxyItemize}
\item Python
\item Java\-Script
\item C++
\end{DoxyItemize}

Under the \href{https://github.com/rapp-project/rapp-api}{\tt rapp-\/api repository} you can find more information regarding implemented and tested Rapp-\/\-Platform Service calls.

As the integration tests are written in Python and uses the \href{https://github.com/rapp-project/rapp-api/tree/master/python}{\tt Python-\/\-Rapp-\/\-A\-Pi} it is a good practice to start on the python-\/rapp-\/api.

\subsubsection*{Python Rapp-\/\-A\-P\-I Overview}

The Python Rapp-\/\-A\-P\-I package can be found under the rapp-\/api/python/ directory, or \href{https://github.com/rapp-project/rapp-api/tree/master/python}{\tt here}.

The package includes the following components\-:


\begin{DoxyItemize}
\item Rapp\-Cloud.\-py\-: For each developed H\-O\-P Web Service this class includes a method that is used to call the relevant H\-O\-P Web Service.
\item Cloud\-Interface\-: Service controller that handles H\-O\-P Web Service calls. It is actually an implementation of an H\-T\-T\-P.\-Post service controller.
\item Configuration files\-: Configuration files which includes information on\-:
\begin{DoxyItemize}
\item Platform authentication credentials, \href{https://github.com/rapp-project/rapp-api/blob/master/python/RappCloud/config/auth.cfg}{\tt auth.\-cfg}. The default user for authentication is the {\bfseries rapp} user. Platform-\/side authentication is handled by the H\-O\-P Server!!
\item Platform address information, \href{https://github.com/rapp-project/rapp-api/blob/master/python/RappCloud/config/platform.cfg}{\tt platform.\-cfg}. Here both the Platform ipv4-\/address and H\-O\-P Server listening port are declared.
\item Platform H\-O\-P Web Services information, \href{https://github.com/rapp-project/rapp-api/blob/master/python/RappCloud/config/services.yaml}{\tt services.\-yaml}. Each developed H\-O\-P Web Service has its own entry into this file.
\end{DoxyItemize}
\end{DoxyItemize}

To extend the Python-\/\-Rapp-\/\-A\-P\-I with a new H\-O\-P Web Service call entry you have to\-:


\begin{DoxyItemize}
\item Implement the specific Rapp\-Cloud class member function.
\item Append the relevant H\-O\-P Web Service name into the {\bfseries services.\-yaml} file.
\end{DoxyItemize}

\subsubsection*{Implement a Rapp\-Cloud class member function as a H\-O\-P Web Service A\-P\-I call.}

Lets say we want to implement the face\-\_\-detection A\-P\-I call. The face\-\_\-detection H\-O\-P Web Service has one input parameter that is a file (image frame).

We implement a member function, named face\-\_\-detection with one input parameter, named {\bfseries file\-\_\-uri}. This parameter is the system path to an image file we want to perform face\-\_\-detection on.

```python class Rapp\-Cloud\-: ... ...

def face\-\_\-detection(self, file\-Uri)\-: pass ```

As you can see the above code just declares the A\-P\-I method and does nothing.

H\-T\-T\-P .post requests can have the following fields\-:


\begin{DoxyItemize}
\item payload\-: The payload declares simple \{parameter, value\} pairs (param1\-: value).
\item files\-: Used to transfer files over the Web, using multipart/form-\/data post requests.
\end{DoxyItemize}

So lets declare two variables into our code to describe the aforementioned fields.

```python def face\-\_\-detection(self, file\-Uri)\-: payload = \{\} files = \{\}

```

The face\-\_\-detection H\-O\-P Web Service has one input parameter, named file\-\_\-uri. We have to declare the {\bfseries file\-\_\-uri} field and append data into that field. Though keep in mind that the {\bfseries file\-\_\-uri} field holds the name of the trasfered, to the Platform H\-O\-P Web Service, file. To ensure that H\-O\-P Service incoming requests will never have conflicts on file\-\_\-names because if this happens files on Platform will overwrite each other, we append a random identity as the file's name postfix ( {\bfseries image1.\-png --$>$ image1-\/akji12.\-png} ). The Rapp\-Cloud class holds an instance of the \href{https://github.com/rapp-project/rapp-api/blob/master/python/RappCloud/RandStrGen/RandStrGen.py}{\tt Random String Generator} class to perform these kind of operations. To append a unique postfix to the file's name call the $\ast$$\ast$\-\_\-\-\_\-append\-Rand\-Str$\ast$$\ast$ member\-:

```python def face\-\_\-detection(self, file\-Uri)\-: file\-Name = self.\-\_\-\-\_\-append\-Rand\-Str(file\-Uri) payload = \{\} files = \{ 'file\-\_\-uri'\-: (file\-Name, open(file\-Uri, 'rb')) \}

```

Notice that we leave the payload to be an empty field!!!

Finally we need call the Cloud\-Interface service controller. Rapp\-Cloud class holds an instance of the Cloud\-Interface class.

```python def face\-\_\-detection(self, file\-Uri)\-: file\-Name = self.\-\_\-\-\_\-append\-Rand\-Str(file\-Uri) payload = \{\} files = \{ 'file\-\_\-uri'\-: (file\-Name, open(file\-Uri, 'rb')) \}

url = self.\-service\-Url\-\_\-\mbox{[}'face\-\_\-detection'\mbox{]}

return\-Data = Cloud\-Interface.\-call\-Service(url, payload, files, self.\-auth\-\_\-) return return\-Data

```

Though one last step is required in order to make the above code execute correctly We need to add the Face-\/\-Detection H\-O\-P Web Service name into the \href{https://github.com/rapp-project/rapp-api/blob/master/python/RappCloud/config/services.yaml}{\tt services.\-yaml} file.

```yaml

services\-: list\-:
\begin{DoxyItemize}
\item face\-\_\-detection
\end{DoxyItemize}

... ```

Make sure to check the Platform Authentication credentials (\href{https://github.com/rapp-project/rapp-api/blob/master/python/RappCloud/config/auth.cfg}{\tt auth.\-cfg}) and Platform Parameters (\href{https://github.com/rapp-project/rapp-api/blob/master/python/RappCloud/config/platform.cfg}{\tt platform.\-cfg}).

By default, Platform Parameters point to the locally installed R\-A\-P\-P-\/\-Platform. Read the {\bfseries platfrom.\-cfg} file on how to change these parameters, if for example you installed the R\-A\-P\-P-\/\-Platform on a remote machine. 