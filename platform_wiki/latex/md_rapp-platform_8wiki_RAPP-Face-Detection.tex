In the R\-A\-P\-P case, the face detection functionality is implemented in the form of a C++ developed R\-O\-S node, interfaced by a H\-O\-P service. The H\-O\-P service is invoked using the R\-A\-P\-P A\-P\-I and gets an R\-G\-B image as input, in which faces must be detected. The second step is for the H\-O\-P service to locally save the input image. At the same time, the Face Detection R\-O\-S node is executed in the background, waiting to server requests. The H\-O\-P service calls the R\-O\-S service via the R\-O\-S Bridge, the R\-O\-S node make the necessary computations and a response is delivered.

\section*{R\-O\-S Services}

\subsection*{Face detection}

Service U\-R\-L\-: {\ttfamily /rapp/rapp\-\_\-face\-\_\-detection/detect\-\_\-faces}

Service type\-: ```bash \#\-Contains info about time and reference Header header \#\-The image's filename to perform face detection \subsection*{string image\-Filename }

\#\-Container for detected face positions geometry\-\_\-msgs/\-Point\-Stamped\mbox{[}\mbox{]} faces\-\_\-up\-\_\-left geometry\-\_\-msgs/\-Point\-Stamped\mbox{[}\mbox{]} faces\-\_\-down\-\_\-right string error ```

\section*{Launchers}

\subsection*{Standard launcher}

Launches the {\bfseries face detection} node and can be launched using ``` roslaunch rapp\-\_\-face\-\_\-detection face\-\_\-detection.\-launch ```

\section*{H\-O\-P services}

\subsection*{U\-R\-L}

{\ttfamily localhost\-:9001/hop/face\-\_\-detection}

\subsection*{Input / Output}

``` Input = \{ “image”\-: “\-T\-H\-E\-\_\-\-A\-C\-T\-U\-A\-L\-\_\-\-I\-M\-A\-G\-E\-\_\-\-D\-A\-T\-A” \} {\ttfamily  } Output = \{ “faces”\-: \mbox{[} \{ “top\-\_\-left\-\_\-x” \-: “\-T\-\_\-\-P\-\_\-\-X”, “top\-\_\-left\-\_\-y” \-: “\-T\-\_\-\-P\-\_\-\-Y”, “bottom\-\_\-right\-\_\-x” \-: “\-B\-\_\-\-R\-\_\-\-X”, “bottom\-\_\-right\-\_\-y” \-: “\-B\-\_\-\-R\-\_\-\-Y” \}, \{ … \} \mbox{]} \} ``` 