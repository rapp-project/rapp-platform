The R\-A\-P\-P Cognitive exercise system aims to provide the Robot users a means of performing basic cognitive exercises. The cognitive tests supported belong to three distinct categories, a) Arithmetic, b) Reasoning/\-Recall, c) Awareness. A number of subcategories exist within each category. Tests have been implemented for all subcategories in different variations and difficulty settings. The N\-A\-O robot is used in order to dictate the test questions to the user and record the answers. A user performance history is being kept in the ontology that aids in keeping track of the user’s cognitive test performance and in adjusting the difficulty of the tests he is presented with. Based on past performance the test difficulty adapts to the user’s specific needs in each test category separately in order to accurately reflect the user’s individual cognitive strengths and weaknesses. To preserve the user’s interest different tests are selected for each category using a least recently used model. To further enhance variation, even tests of the same subcategory exist in different variations. The R\-A\-P\-P Cognitive exercise system component diagram is depicted below.

\mbox{[}\mbox{[}images/cognitive\-\_\-exercise\-\_\-system\-\_\-component\-\_\-diagram.\-png\mbox{]}\mbox{]}

\section*{R\-O\-S Services}

\subsection*{Test Selector}

This service was created in order to select the appropriate test for a user given its type. The service will load the user’s past performance from the ontology, determine the appropriate difficulty setting for the specific user and return the least recently used test of its type. In case no test type is provided then the least recently used test type category will be selected.

Service U\-R\-L\-: {\ttfamily /rapp/rapp\-\_\-cognitive\-\_\-exercise/cognitive\-\_\-exercise\-\_\-chooser}

Service type\-: ```bash \#\-Contains info about time and reference Header header \#\-The username of the user string username \#\-The test type requested \subsection*{String test\-Type }

\#\-The test’s name string test \#\-The test’s type string test\-Type \#\-The test’s sub type string test\-Sub\-Type \#\-The test’s language string language \#\-The test’s questions string\mbox{[}\mbox{]} questions \#\-The list of answers for each test string\mbox{[}\mbox{]}\mbox{[}\mbox{]} answers \#\-The correct answers for each test string\mbox{[}\mbox{]} correct\-\_\-answers \#\-Possible error string error ```

\subsection*{Record User Performance}

This service was created in order to record the performance, meaning the score, of a user after completing a cognitive exercise test. The name of the test, its type and the user’s score are provided as input arguments. Within the service, a Unix timestamp is automatically generated. The timestamp reflects the time at which the test was performed. The service returns the name of the test performance entry that was created in the ontology.

Service U\-R\-L\-: {\ttfamily /rapp/rapp\-\_\-cognitive\-\_\-exercise/record\-\_\-user\-\_\-cognitive\-\_\-test\-\_\-performance}

Service type\-: ```bash \#\-Contains info about time and reference Header header \#\-The username of the user string username \#\-The type of the test string test\-Type \#\-The test which was performed string test \#\-The score the user achieved on the test \subsection*{String score }

\#\-Container for the subclasses String user\-\_\-cognitive\-\_\-test\-\_\-performance\-\_\-entry \#\-Possible error string error ```

\subsection*{Cognitive Test creator}

This service was created in order to create a cognitive test in the special xml format required and register it to the ontology. It accepts as an input a specially formatted text file containing all the information required for the test including its type, subtype, difficulty, questions, answers etc. The service formats the above information in an xml file and registers the test along with some vital information like it’s type and difficulty setting to the ontology. The service returns a bool variable that is true if xml creation and ontology registration were successful. Any possible error is contained in the error string variable also returned.

Service U\-R\-L\-: {\ttfamily /rapp/rapp\-\_\-cognitive\-\_\-exercise/cognitive\-\_\-test\-\_\-creator}

Service type\-: ```bash \#\-Contains info about time and reference Header header \#\-The text file containing the test information \subsection*{string input\-File }

\#\-True if test creation and registration to the ontology was successfull bool success \#\-Possible error string error ```

\section*{Launchers}

\subsection*{Standard launcher}

Launches the {\bfseries rapp\-\_\-cognitive\-\_\-exercise} node and can be launched using ``` roslaunch rapp\-\_\-cognitive\-\_\-exercise cognitive\-\_\-exercise.\-launch ```

\section*{H\-O\-P services}

\subsection*{Test Selector R\-P\-S}

The test\-\_\-selector R\-P\-S is of type 3 since it contains a H\-O\-P service front-\/end, contacting a R\-A\-P\-P R\-O\-S ontology wrapper, which performs queries to the Know\-Rob ontology repository. The get subclass of R\-P\-S can be invoked using the following U\-R\-L.

Service U\-R\-L\-: {\ttfamily localhost\-:9001/hop/test\-\_\-selector}

\subsubsection*{Input/\-Output}

The test\-\_\-selector R\-P\-S has two arguments, which are the username of the user and the requested test type. This is encoded in J\-S\-O\-N format in an A\-S\-C\-I\-I string representation.

The test\-\_\-selector R\-P\-S the questions, the possible answers and the correct answers of the selected test. The encoding is in J\-S\-O\-N format.

``` Input = \{ “username”\-: “\-T\-H\-E\-\_\-\-U\-S\-E\-R\-N\-A\-M\-E” “test\-Type”\-: “\-T\-H\-E\-\_\-\-T\-E\-S\-T\-\_\-\-T\-Y\-P\-E” \} {\ttfamily  } Output = \{ “test”\-: “\-The test’s name” “test\-Type”\-: “\-The test’s type” “test\-Sub\-Type”\-: “\-The test’s sub type” “language”\-: “\-The test’s language” “questions”\-: “\-The questions of the test” “answers”\-: “\-The possible answers for each question of the test” “correct\-\_\-answers”\-: “\-The correct answer for each question of the test” “error”\-: “\-Possible error” \}

``` \subsubsection*{Example}

An example input for the test\-\_\-selector R\-P\-S is ``` Input = \{ “instance\-\_\-name”\-: “\-Person\-\_\-1” “attribute\-\_\-names”\-: “\-Arithmetic\-Cts” \} ```

For this specific input, the result obtained was

``` Output = \{ “test”\-: “\-Arithmetic\-Cts\-\_\-\-X\-X\-X\-X\-X\-X\-X” “test\-Type”\-: “\-Arithmeti\-Cts” “test\-Sub\-Type”\-: “\-Basic\-Arithmetic” “language”\-: “en” “questions”\-: “\mbox{[}How much is 2 plus 2?, How much is 5 plus 5?\mbox{]}” “answers”\-: “\mbox{[}2,3,4,5\mbox{]},\mbox{[}7,8,9,10\mbox{]}” “correct\-\_\-answers”\-: “\mbox{[}4,10\mbox{]}" \} ```

\subsection*{Record user cognitive test performance R\-P\-S}

The record\-\_\-user\-\_\-cognitive\-\_\-test\-\_\-performance R\-P\-S is of type 3 since it contains a H\-O\-P service frontend, contacting a R\-A\-P\-P R\-O\-S ontology wrapper, which performs queries to the Know\-Rob ontology repository. The record user cognitive test performance of R\-P\-S can be invoked using the following U\-R\-L.

Service U\-R\-L\-: {\ttfamily localhost\-:9001/hop/record\-\_\-user\-\_\-cognitive\-\_\-test\-\_\-performance}

\subsubsection*{Input/\-Output}

The record\-\_\-user\-\_\-cognitive\-\_\-test\-\_\-performance R\-P\-S has four arguments, which are the username of the user the test which was taken, the test’s type and the score achieved. This is encoded in J\-S\-O\-N format in an A\-S\-C\-I\-I string representation.

The record\-\_\-user\-\_\-cognitive\-\_\-test\-\_\-performance R\-P\-S outputs only a possible error message which is empty when the query was successful. The encoding is in J\-S\-O\-N format.

``` Input = \{ “username”\-: “\-T\-H\-E\-\_\-\-U\-S\-E\-R\-N\-A\-M\-E” “test”\-: “\-T\-H\-E\-\_\-\-T\-E\-S\-T” “test\-Type”\-: “\-T\-H\-E\-\_\-\-T\-E\-S\-T\-\_\-\-T\-Y\-P\-E” \} {\ttfamily  } Output = \{ “error”\-: “\-Possible error” \} ```

\subsubsection*{Example}

An example input for the record\-\_\-user\-\_\-cognitive\-\_\-test\-\_\-performance R\-P\-S is ``` Input = \{ “instance\-\_\-name”\-: “\-Person\-\_\-1” “test”\-: “\-Test1” “test\-Type”\-: “\-Arithmetic\-Cts” “score”\-: “90” \} ``` For this specific input, the result obtained was

``` Output = \{ “error”\-: ``` 