The overall design of the R\-A\-P\-P Architecture is depicted in the following figure.

\mbox{[}\mbox{[}images/rapp\-\_\-platform\-\_\-architecture.\-png\mbox{]}\mbox{]}

As evident, the R\-A\-P\-P ecosystem consists of two major components\-: the R\-A\-P\-P Platform (upper half) and the respective Robot Platform (bottom half). These systems are not only functionally, but also physically separated, as the Platform resides in the cloud and the Robot Platform obviously represents each robot supported by the R\-A\-P\-P system, constituting our architecture two layered.

The upper layer is the R\-A\-P\-P ecosystem’s cloud part, consisting of three main parts\-: The R\-A\-P\-P Store, the Platform Agent and a R\-App’s Cloud agent. The R\-A\-P\-P Store provides the front-\/end of the application store, allowing the creation of accounts either from the robotic application developers or from the end users. Once a developer creates his account he/she is able to submit R\-Apps via an interface, which are cross-\/compiled for the supported robots, packaged and indexed. Then they can be distributed at the corresponding robots. If any errors occur during this procedure, the developer is informed in order to correct them and resubmit the application. On the other hand, the end users are able to log in and select the R\-Apps they desire for their robot to execute.

The main part of the R\-A\-P\-P Platform is the R\-A\-P\-P Platform Agent, where the core of the developed artificial intelligence resides. This includes a My\-S\-Q\-L database, where all the eponymous information is safely being stored, offline learning procedures and the R\-A\-P\-P Improvement Centre (R\-I\-C). This constitutes of the ontology knowledge base, machine learning algorithms and various robotic-\/oriented algorithms. These are either developed or wrapped by R\-O\-S nodes. Finally, these algorithms can be utilized by robots via the H\-O\-P services exposed by R\-I\-C. The H\-O\-P services can be invoked by the R\-A\-P\-P A\-P\-I with specific arguments, providing access uniformity and authentication security. The final R\-A\-P\-P Platform part is the Cloud Agent, which will be described later for better comprehension reasons.

The lower layer of the R\-A\-P\-P architecture is the Robot Platforms. There three specific components exist\-: The Core agent, the Dynamic agent and the Communication layer. The Core agent is robot specific and is provided by the R\-A\-P\-P platform. It uptakes the tasks of downloading / updating the R\-Apps and providing robot-\/specific services to the applications. These usually are high level interfaces to the robot’s hardware or to locally embedded algorithms, such as mapping, navigation or path planning. A robot can utilize either low level drivers for interfacing with its hardware or higher level tools, such as R\-O\-S2. When a R\-App is downloaded and executed by the Core agent, a Dynamic agent is created, essentially being another naming for the R\-App’s robot part. The Dynamic agent can be a R\-O\-S package, a Java\-Script file, or a combination of them. This uses the R\-A\-P\-P A\-P\-I to invoke services either on the Platform or in the robot. One important aspect of the overall architecture is that the developer may choose for his application to be executed in a decentralized way, meaning that according to his submission, one part of the code may execute in the robot and another part of the cloud. The second part is the Cloud agent described before, which is initiated and killed synchronously to the Dynamic agent.

Considering the R\-A\-P\-P database, it resides on the R\-A\-P\-P Platform and can be accessed a) by the R\-A\-P\-P Store, in order to keep track of the user accounts and the submitted / downloaded applications and b) by the R\-I\-C for machine learning / statistical purposes and data acquisition. The latter is performed via a R\-O\-S My\-S\-Q\-L wrapper, providing interfaces to all the nodes in need for stored data.

Regarding the semantic / knowledge part of R\-A\-P\-P, we decided to employ already used and tested ontologies, aiming at not reinventing the wheel. The only limitations towards this selection were that the ontologies a) should have R\-O\-S support, since the R\-A\-P\-P Platform will be R\-O\-S-\/based, b) should have a common file format to easily merge the information and c) to be state-\/of-\/the-\/art in their respective fields. Since R\-A\-P\-P is multidisciplinary, the selected ontologies contain concepts from heterogeneous scientific fields. After researching in the ontology domain the Know\-Rob and Open\-A\-A\-L ontologies were selected for deployment.

Know\-Rob was part of the Robo\-Earth F\-P7 project and was specially designed to be used by robots, extending classical ontologies and concepts in a machine-\/usable way. Know\-Rob is a state-\/of-\/the-\/art robotic ontology, used by many scientific teams in cloud robotics projects for concepts storage, inference and distribution of knowledge. One great advantage to our cause is that bindings to R\-O\-S exist, as it would be challenging to utilize it otherwise, being developed in S\-W\-I-\/\-Prolog. Finally, The R\-O\-S bindings are created using the J\-P\-L interface (Java / Prolog bidirectional Interface) and the rosjava package. Finally, Know\-Rob provides an abundance of ontology packages, all encoded in O\-W\-L files that can be dynamically loaded in the Know\-Rob software tool and queried. On the other hand, regarding the ambient assisted living field, the Open\-A\-A\-L ontology was selected, created for the S\-O\-P\-R\-A\-N\-O integrated project, providing a middleware for A\-A\-L scenarios. It should be stated that similarly to Know\-Rob, Open\-A\-A\-L is not just an ontology but a software tool providing methods for context management, multi-\/paradigm context augmentation, context aware behaviours and others. In our case we will not utilize the overall software package but only the ontology semantic information, encoded in O\-W\-L files.

As obvious, Know\-Rob and Open\-A\-A\-L are two different ontologies, intended for heterogeneous applications, thus means to utilize them efficiently and meaningfully must be implemented. An initial thought is to load both O\-W\-L files in the Know\-Rob software tool, in order to be able to access the information via R\-O\-S interfaces. Even though this would be possible, no higher level semantic processes can be deployed, since the two ontology upper level taxonomies would be semantically separated. Thus we decided to semantically merge the two taxonomies and to enrich them with classes relevant to the R\-A\-P\-P’s scope. Regarding the R\-A\-P\-P ontology’s access a Know\-Rob R\-O\-S wrapper was created that will provide querying capabilities to external services.

The R\-A\-P\-P Platform component diagram is evident in the image below.

\mbox{[}\mbox{[}images/rapp\-\_\-platform\-\_\-class\-\_\-diagram.\-png\mbox{]}\mbox{]} 