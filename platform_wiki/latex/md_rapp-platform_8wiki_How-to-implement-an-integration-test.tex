It is recommended to study on \href{https://github.com/rapp-project/rapp-platform/wiki/RAPP-Testing-Tools}{\tt Rapp-\/\-Testing-\/\-Tools} first.

Basic documentation on \char`\"{}$\ast$$\ast$\-Developing Integration Tests$\ast$$\ast$\char`\"{} can be found \href{https://github.com/rapp-project/rapp-platform/tree/master/rapp_testing_tools}{\tt here}.

A more-\/in-\/depth information on how to write your first integration test is presented here.

The \href{https://github.com/rapp-project/rapp-platform/blob/master/rapp_testing_tools/scripts/default_tests/template_test.py}{\tt template\-\_\-test.\-py} will be used as a reference.

Each integration test must have the following characteristics\-:


\begin{DoxyItemize}
\item Each test class is written as a seperate python source file (.py).
\item Each test inherits from {\ttfamily unittest.\-Test\-Case} class.
\item The \href{https://github.com/rapp-project/rapp-api/tree/master/python}{\tt Python R\-A\-P\-P Platform A\-P\-I} is used to call R\-A\-P\-P Platform Services.
\end{DoxyItemize}

If you are not familiar with the Python unit testing framework (unittest), start reading from \href{https://docs.python.org/2.7/library/unittest.html}{\tt there} as the rapp\-\_\-testing\-\_\-tools is build ontop of it. Also, basic knowledge of using the \href{https://github.com/rapp-project/rapp-api/tree/master/python}{\tt python-\/rapp-\/platform-\/api} is required.

So lets create a test class for testing the integration behaviour of the Face-\/\-Detection R\-A\-P\-P Platform service/functionality.

\paragraph*{Step 1\-: Create the source file for the test class}

Copy the {\ttfamily template\-\_\-test.\-py} file and name it {\ttfamily face\-\_\-detection\-\_\-tests.\-py} under the {\ttfamily default\-\_\-tests} directory

```bash \$ cd $\sim$/rapp\-\_\-platform/rapp-\/platform-\/catkin-\/ws/src/rapp-\/platform/rapp\-\_\-testing\-\_\-tools/scripts/default\-\_\-tests \$ cp template\-\_\-test.\-py face\-\_\-detection\-\_\-tests.\-py ```

Rename the test class to {\ttfamily Face\-Detection\-Tests}\-:

```python class Face\-Detection\-Tests(unittest.\-Test\-Case)\-: \begin{DoxyVerb}def setUp(self):
    self.ch = RappPlatformAPI()

    rospack = rospkg.RosPack()
    self.pkgDir = rospack.get_path('rapp_testing_tools')

def test_templateTest(self):
    self.assertEqual(1, 1)
\end{DoxyVerb}


```

Note that the {\ttfamily set\-Up()} method instantiates a Rapp\-Platform\-A\-P\-I() object for calling R\-A\-P\-P Platform Services.

\paragraph*{Step 2\-: Implement a test case}

As the {\ttfamily Face\-Detection\-Tests} class inherits from {\ttfamily unittest.\-Test\-Case}, individual tests are defined with methods whose names start with the letters {\ttfamily test}

We will implement a test of performing face-\/detection on a single image file (Lenna.\-png). We assume that the image file was previously stored under the {\ttfamily test\-\_\-data} directory. So lets create a member method named {\ttfamily test\-\_\-lenna} This method will be responsible for loading the image file, call the R\-A\-P\-P Platform Service, through the A\-P\-I, and evaluate the results using {\ttfamily unittest} assertions

```python class Face\-Detection\-Tests(unittest.\-Test\-Case)\-: \begin{DoxyVerb}def setUp(self):
    self.ch = RappPlatformAPI()

    rospack = rospkg.RosPack()
    self.pkgDir = rospack.get_path('rapp_testing_tools')

def test_lenna(self):
    self.assertEqual(1, 1)
    response = self.ch.faceDetection(imagepath)

    valid_faces = [{
        'up_left_point': {'y': 201.0, 'x': 213.0},
        'down_right_point': {'y': 378.0, 'x': 390.0}
    }]

    self.assertEqual(response['error'], u'')
    self.assertEqual(response['faces'], valid_faces)
\end{DoxyVerb}
 ```

The first assertion, `self.assert\-Equal(response\mbox{[}'error'\mbox{]}, u'')` evaluates that no error was reported from the R\-A\-P\-P Platform The second assertion evaluates the response from the Face\-Detector.

The complete face\-\_\-detection\-\_\-tests.\-py source file is\-:

```python from os import path import timeit import unittest import rospkg

{\bfseries path} = os.\-path.\-dirname(os.\-path.\-realpath({\bfseries file}))

from Rapp\-Cloud import Rapp\-Platform\-A\-P\-I

class Face\-Detection\-Tests(unittest.\-Test\-Case)\-: \begin{DoxyVerb}def setUp(self):
    self.ch = RappPlatformAPI()

    rospack = rospkg.RosPack()
    self.pkgDir = rospack.get_path('rapp_testing_tools')

def test_lenna(self):
    self.assertEqual(1, 1)
    response = self.ch.faceDetection(imagepath)

    valid_faces = [{
        'up_left_point': {'y': 201.0, 'x': 213.0},
        'down_right_point': {'y': 378.0, 'x': 390.0}
    }]

    self.assertEqual(response['error'], u'')
    self.assertEqual(response['faces'], valid_faces)
\end{DoxyVerb}


if {\bfseries name} == \char`\"{}\-\_\-\-\_\-main\-\_\-\-\_\-\char`\"{}\-: unittest.\-main() ```

\paragraph*{Step 3\-: Executing the Face Detection test case}

Head to the {\ttfamily scripts} directory and execute the {\ttfamily rapp\-\_\-run\-\_\-test.\-py} script, giving as input argument the {\ttfamily face\-\_\-detection\-\_\-tests.\-py} file to execute\-:

```bash \$ cd $\sim$/rapp\-\_\-platform/rapp-\/platform-\/catkin-\/ws/src/rapp-\/platform/rapp\-\_\-testing\-\_\-tools/scripts \$ ./rapp\-\_\-run\-\_\-test.py -\/i default\-\_\-tests/face\-\_\-detection\-\_\-tests.\-py ```

On successful test execution, the output should be\-:

```bash 

 R\-A\-P\-P Platfrom Tests 




\begin{DoxyItemize}
\item Parameters\-: -- Number of Executions for each given test\-: \mbox{[}1\mbox{]} -- Serial execution
\item Tests to Execute\-: 1\mbox{]} face\-\_\-detection\-\_\-tests x1
\end{DoxyItemize}

Running face\-\_\-detection\-\_\-tests... Ran 1 tests in 0.\-383s Success ```

If an error occures on the R\-A\-P\-P Platform, the returned error message will be reported to the console output. For example, if the R\-A\-P\-P Platform Web Server was not previously launched properly, we should get a \char`\"{}\-Connection\char`\"{} error on the assertion of the {\ttfamily error} property of the response\-:

```bash 

 R\-A\-P\-P Platfrom Tests 




\begin{DoxyItemize}
\item Parameters\-: -- Number of Executions for each given test\-: \mbox{[}1\mbox{]} -- Serial execution
\item Tests to Execute\-: 1\mbox{]} face\-\_\-detection\-\_\-tests x1
\end{DoxyItemize}

Running face\-\_\-detection\-\_\-tests... Ran 1 test in 0.\-078s Failed 

 \subsection*{F\-A\-I\-L\-: test\-\_\-lenna ({\bfseries main}.Face\-Detection\-Tests) }

Traceback (most recent call last)\-: File \char`\"{}/home/rappuser/rapp\-\_\-platform/rapp-\/platform-\/catkin-\/ws/src/rapp-\/platform/rapp\-\_\-testing\-\_\-tools/scripts/default\-\_\-tests/face\-\_\-detection\-\_\-tests.\-py\char`\"{}, line 61, in test\-\_\-lenna self.\-assert\-Equal(response\mbox{[}'error'\mbox{]}, u'') Assertion\-Error\-: 'Connection Error' != u'' 

 Ran 1 test in 0.\-078s

F\-A\-I\-L\-E\-D (failures=1)

``` 