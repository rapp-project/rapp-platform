R\-A\-P\-P Web Services are implemented on top of the \href{http://hop.inria.fr/home/index.html}{\tt hop.\-js} framework. \begin{quotation}
Hop.\-js is a multitier extension of Java\-Script. It allows a single Java\-Script program to describe the client-\/side and the server-\/side components of a Web application.

\end{quotation}
Its runtime environment ensures a consistent execution of the application on the server and on the client. Hop programs execute in the context of a builtin web server. They define services, which are super Java\-Script functions that get automatically invoked when H\-T\-T\-P requests are received.

A framework has been developed, build ontop of hop.\-js, for easily implementing Web Services for the R\-A\-P\-P Platform. Zero knowledge of hop.\-js is required. Additionally, Web Services are fully parametrized through single configuration files\-:


\begin{DoxyItemize}
\item \href{https://github.com/rapp-project/rapp-platform/blob/master/rapp_web_services/config/services/services.json}{\tt services.\-json}
\item \href{https://github.com/rapp-project/rapp-platform/blob/master/rapp_web_services/config/services/workers.json}{\tt workers.\-json}
\end{DoxyItemize}

If you are not familiar with server-\/side applications, you might also have to read on \href{https://nodejs.org/en/}{\tt Nodejs}.

\subsection*{R\-A\-P\-P Web Services framework}

For easily implementing and launching Web Services for the R\-A\-P\-P Platform, the R\-A\-P\-P Web Services framework has been developed. It consists of a \href{https://github.com/rapp-project/rapp-platform/tree/master/rapp_web_services/src/webService}{\tt Web\-Service} implementation. build ontop of hop.\-js and an engine that allows to assign specific Web Services to Worker threads (Web Workers).

\subsubsection*{Web Service implementation}

Web Services are implemented in a single function implementation\-:

```js function svc\-Impl(req, resp, ros)\{ // Web service implementation here. \} ```

This is the onrequest callback function to feed to the engine and will be called as soon as a request arrives.

\begin{quotation}
The {\bfseries req} (request) and {\bfseries res} (response) objects are passed so you can access the request properties and craft and return responses. Additionally a {\bfseries ros} object is passed that allows connections to R\-O\-S thought the rosbridge-\/websocket transport layer.

\end{quotation}


The {\ttfamily req} object has the following properties\-:
\begin{DoxyItemize}
\item {\ttfamily header}\-: Request header.
\end{DoxyItemize}

```js console.\-log(req.\-header) \begin{quotation}
\begin{quotation}
\{ host\-: 'localhost\-:9001', content-\/length\-: 679, accept-\/encoding\-: 'gzip, deflate', accept\-: '$\ast$/$\ast$', user-\/agent\-: 'rapp-\/platform-\/api/python', accept-\/token\-: 'rapp\-\_\-token', connection\-: 'keep-\/alive', content-\/type\-: 'multipart/form-\/data; boundary=595d1046de7f4c958ca662c37140215a' \}

\end{quotation}


\end{quotation}
```


\begin{DoxyItemize}
\item {\ttfamily socket}\-: Connection socket
\end{DoxyItemize}

```js console.\-log(req.\-socket) \begin{quotation}
\begin{quotation}
\{ hostname\-: 'localhost', host\-Address\-: '127.\-0.\-0.\-1', local\-Address\-: '127.\-0.\-0.\-1', port\-: 43661 \}

\end{quotation}


\end{quotation}
```


\begin{DoxyItemize}
\item {\ttfamily body}\-: Request body
\end{DoxyItemize}

```js console.\-log(req.\-body) \begin{quotation}
\begin{quotation}
\{ fast\-: true \}

\end{quotation}


\end{quotation}
```


\begin{DoxyItemize}
\item {\ttfamily files}\-: In case of uploading files, this field contains the paths to the uploaded files. Access the files by name using dot notation. For example a service receives a single-\/file in fieldname {\ttfamily single\-\_\-file} and an array of files in fieldname {\ttfamily file\-\_\-array}\-:
\end{DoxyItemize}

```js console.\-log(req.\-files) \begin{quotation}
\begin{quotation}
\{ single\-\_\-file\-: \mbox{[}\char`\"{}\-P\-A\-T\-H\char`\"{}\mbox{]}, file\-\_\-array\-: \mbox{[}\char`\"{}\-P\-A\-T\-H\-\_\-\-F\-I\-L\-E\-\_\-1\char`\"{}, \char`\"{}\-P\-A\-T\-H\-\_\-\-F\-I\-L\-E\-\_\-2\char`\"{}\mbox{]} \}

\end{quotation}


\end{quotation}
```

{\bfseries Note}\-: Note that even if it is a {\bfseries single-\/file}, the {\ttfamily single\-\_\-file} property of {\ttfamily req.\-files} ({\ttfamily req.\-files.\-single\-\_\-file}) is an Array.


\begin{DoxyItemize}
\item {\ttfamily username}\-: This is the {\bfseries username} of the client that requested access to the R\-A\-P\-P Platform resources (Services). It is automatically applied to the {\ttfamily req} object, after appliance of the \href{https://github.com/rapp-project/rapp-platform/wiki/RAPP-Application-Authentication}{\tt R\-A\-P\-P Authentication} on request arrival. Note that before execution of the onrequest callback function, we apply authentication to the request. If the authentication is not successful, an {\bfseries H\-T\-T\-P 401 Unauthorized} error is returned to the client.
\end{DoxyItemize}

```js console.\-log(req.\-username) \begin{quotation}
\begin{quotation}
\char`\"{}\-R\-A\-P\-P\-\_\-\-U\-S\-E\-R\char`\"{}

\end{quotation}


\end{quotation}
```

The {\ttfamily resp} object has the following properties (methods)\-:


\begin{DoxyItemize}
\item send\-Json(obj)\-: Send an application/json response
\end{DoxyItemize}

```js function svc\-Impl(req, resp, ros)\{ ...

var response = \{error\-: ''\}; resp.\-send\-Json(response); \} ```


\begin{DoxyItemize}
\item {\ttfamily send\-Server\-Error()}\-: Respond with {\bfseries H\-T\-T\-P 500 Internal Server Error}
\end{DoxyItemize}

```js function svc\-Impl(req, resp, ros)\{ ...

resp.\-send\-Server\-Error(); \} ```


\begin{DoxyItemize}
\item {\ttfamily send\-Unauthorized()}\-: Respond with {\bfseries H\-T\-T\-P 401 Unauthorized Client Error}
\end{DoxyItemize}

```js function svc\-Impl(req, resp, ros)\{ ...

resp.\-send\-Unauthorized(); \} ```

\subsubsection*{Web Service configuration and registration.}

Web Services are fully parametrized through the \href{https://github.com/rapp-project/rapp-platform/blob/master/rapp_web_services/config/services/services.json}{\tt services.\-json} file. This file includes Web Services to be launched (along with the Web Service parameters), that was previously declared in the \href{https://github.com/rapp-project/rapp-platform/blob/master/rapp_web_services/config/services/workers.json}{\tt workers.\-json} file.

Below is the face\-\_\-detection entry\-:

```js \char`\"{}face\-\_\-detection\char`\"{}\-: \{ \char`\"{}launch\char`\"{}\-: true, \char`\"{}anonymous\char`\"{}\-: false, \char`\"{}name\char`\"{}\-: \char`\"{}face\-\_\-detection\char`\"{}, \char`\"{}url\-\_\-name\char`\"{}\-: \char`\"{}face\-\_\-detection\char`\"{}, \char`\"{}namespace\char`\"{}\-: \char`\"{}\char`\"{}, \char`\"{}ros\-\_\-connection\char`\"{}\-: true, \char`\"{}timeout\char`\"{}\-: 45000 \} ```

{\bfseries Web Service configuration parameters}\-:


\begin{DoxyItemize}
\item {\ttfamily launch} (Boolean)\-: If true this Web Service will be launched.
\item {\ttfamily anonymous} (Boolean)\-: If true, this service will be anonymous, which means that it will be assigned a random url path.
\item {\ttfamily name} (String)\-: The service name.
\item {\ttfamily urlname} (String)\-: The service urlname. Service name can be different from the urlname.
\item {\ttfamily namespace} (String)\-: Namespace for the urlname to append as a prefix to the service url name. For example, a service with {\bfseries urlname=\char`\"{}faca\-\_\-detection\char`\"{}} and {\bfseries namespace=\char`\"{}computervision\char`\"{}} will be translated to $\ast$$\ast$/computervision/face\-\_\-detection$\ast$$\ast$
\item {\ttfamily timeout} (String)\-: Request timeout value.
\item {\ttfamily ros\-\_\-connection} (Boolean)\-: If true, a ros object that allowes for calls to the R\-O\-S Services will be passed to the onrequest callback function.
\end{DoxyItemize}

\subsubsection*{Run a Web Service within an existing Web Worker}

Web services run within server-\/side workers (Web Workers). A worker can include more than one web service. We consider server-\/side workers to be forked processes, thus allowing concurrent execution.

To run a Web Service within a Web Worker, just specify the service name in the {\bfseries services} field of the worker in the \href{https://github.com/rapp-project/rapp-platform/blob/master/rapp_web_services/config/services/workers.json}{\tt worker.\-json} file.

For example, the {\bfseries weather\-\_\-report} worker holds the {\bfseries weather\-\_\-report\-\_\-current} and {\bfseries weather\-\_\-report\-\_\-forecast} Web Services\-:

```js \char`\"{}weather\-\_\-report\char`\"{}\-: \{ \char`\"{}launch\char`\"{}\-: true, \char`\"{}path\char`\"{}\-: \char`\"{}workers/weather\-\_\-report.\-js\char`\"{}, \char`\"{}services\char`\"{}\-: \mbox{[} \char`\"{}weather\-\_\-report\-\_\-forecast\char`\"{}, \char`\"{}weather\-\_\-report\-\_\-current\char`\"{} \mbox{]} \} ```

{\bfseries Web Worker configuration parameters}\-:


\begin{DoxyItemize}
\item {\ttfamily launch} (Boolean)\-: Weather to launch the Web Worker or not.
\item {\ttfamily path} (String)\-: Path to the Web Worker source file. Relative to the {\bfseries rapp\-\_\-web\-\_\-services} directory
\item {\ttfamily services}\-: (Array)\-: Services to launch under the Web Worker thread.
\end{DoxyItemize}

\subsubsection*{Where to store Web Service implementation source file(s) and how to launch it.}

Source files are stored under the \href{https://github.com/rapp-project/rapp-platform/tree/master/rapp_web_services/services}{\tt services} directory, of the \href{https://github.com/rapp-project/rapp-platform/tree/master/rapp_web_services}{\tt rapp\-\_\-web\-\_\-services} package.

Web Services are automatically loaded from single .js files, as node.\-js modules. Make sure you export the Web Service implementation function\-:

```js function svc\-Impl(req, resp, ros)\{ // Web service implementation here. \}

...

module.\-exports = svc\-Impl;

```

\subsubsection*{Complete Web Service Implementation Example -\/ Face Detection}

The following example illustrates the implementation of a Web\-Service that connects to the Face-\/\-Detection R\-A\-P\-P-\/\-Platform backend Service

{\bfseries R\-O\-S Service Message}\-: ```bash \section*{Contains info about time and reference}

Header header \section*{The image's filename to perform face detection}

string image\-Filepath \section*{Flag to define if a fast detection if desired}

\subsection*{bool fast }

\section*{Container for detected face positions}

geometry\-\_\-msgs/\-Point\-Stamped\mbox{[}\mbox{]} faces\-\_\-up\-\_\-left geometry\-\_\-msgs/\-Point\-Stamped\mbox{[}\mbox{]} faces\-\_\-down\-\_\-right string error ```

and the Face-\/\-Detection R\-O\-S Service url path is\-: {\ttfamily /rapp/rapp\-\_\-face\-\_\-detection/detect\-\_\-faces}

{\bfseries Web Service Request}\-:


\begin{DoxyItemize}
\item {\ttfamily file}\-: Image file.
\item {\ttfamily fast} (Bool)\-: If true, detection will take less time but it will be less accurate.
\end{DoxyItemize}

{\bfseries Web Service Response}\-:


\begin{DoxyItemize}
\item {\ttfamily faces} (Array)\-: Detected faces.
\item {\ttfamily error} (String)\-: Error message.
\end{DoxyItemize}

First, create the face\-\_\-detection {\ttfamily svc.\-js} file\-:

```bash \$ cd $\sim$/rapp\-\_\-platform/rapp-\/platform-\/catkin-\/ws/src/rapp-\/platform/rapp\-\_\-web\-\_\-services/services \$ mkdir face\-\_\-detection \&\& cd face\-\_\-detection \$ touch svc.\-js ```

Open the svc.\-js file with your favorite editor\-:

```bash \$ vim svc.\-js ```

Implement the structure of the {\ttfamily client-\/response} and {\ttfamily ros-\/msg} objects\-:

```js var client\-Res = function(faces, error) \{ return \{ faces\-: \mbox{[}\mbox{]}, error\-: '' \} \}

var ros\-Req\-Msg = function(image\-Filepath, fast) \{ return \{ image\-Filepath\-: '', fast\-: false \} \} ```

Assign R\-O\-S Service url path to a global variable\-:

```js var ros\-Srv\-Url\-Path = \char`\"{}/rapp/rapp\-\_\-face\-\_\-detection/detect\-\_\-faces\char`\"{}; ```

Next, implement the service onrequest callback function\-:

```js function svc\-Impl(req, resp, ros) \{ // If no image file received, return to client with an error if( ! req.\-files.\-file )\{ // Create a client response object response = new client\-\_\-res(); response.\-error = \char`\"{}\-No image file received\char`\"{}; // Send response (application/json) resp.\-send\-Json(response); return; \}

// Create a R\-O\-S Service Request Message and fill values from client request var ros\-Msg = new ros\-Req\-Msg(); ros\-Msg.\-image\-Filename = req.\-files.\-file\mbox{[}0\mbox{]}; ros\-Msg.\-fast = req.\-body.\-fast;

/$\ast$$\ast$$\ast$
\begin{DoxyItemize}
\item R\-O\-S-\/\-Service response callback. $\ast$/ function callback(data)\{ // Delete image file from the Platform cache directory. fs.\-exists(\-\_\-filepath, function(exists)\{ if(exists)\{ fs.\-unlink(\-\_\-filepath) \} \}) // Parse rosbridge message and craft client response var response = parse\-Rosbridge\-Msg( data ); resp.\-send\-Json(response); \}
\end{DoxyItemize}

/$\ast$$\ast$$\ast$
\begin{DoxyItemize}
\item R\-O\-S-\/\-Service onerror callback. $\ast$/ function onerror(e)\{ // Delete image file from the Platform cache directory. fs.\-exists(\-\_\-filepath, function(exists)\{ if(exists)\{ fs.\-unlink(\-\_\-filepath) \} \}) // Respond a \char`\"{}\-Server Error\char`\"{}. H\-T\-T\-P Error 501 -\/ Internal Server Error resp.\-send\-Server\-Error(); \}
\end{DoxyItemize}

/$\ast$$\ast$$\ast$
\begin{DoxyItemize}
\item Call R\-O\-S-\/\-Service. $\ast$/ ros.\-call\-Service(ros\-Srv\-Url\-Path, ros\-Msg, \{success\-: callback, fail\-: onerror\}); \}
\end{DoxyItemize}

function parse\-Rosbridge\-Msg( rosbridge\-\_\-msg ) \{ var faces\-\_\-up\-\_\-left = rosbridge\-\_\-msg.\-faces\-\_\-up\-\_\-left; var faces\-\_\-down\-\_\-right = rosbridge\-\_\-msg.\-faces\-\_\-down\-\_\-right; var error = rosbridge\-\_\-msg.\-error; var num\-Faces = faces\-\_\-up\-\_\-left.\-length;

// Create a new response object var response = new client\-\_\-res();

if( error )\{ // If R\-O\-S Service responded with an error response.\-error = error; return response; \}

for (var ii = 0; ii $<$ num\-Faces; ii++) \{ var face = \{ up\-\_\-left\-\_\-point\-: \{x\-: 0, y\-:0\}, down\-\_\-right\-\_\-point\-: \{x\-: 0, y\-: 0\} \};

face.\-up\-\_\-left\-\_\-point.\-x = faces\-\_\-up\-\_\-left\mbox{[}ii\mbox{]}.point.\-x; face.\-up\-\_\-left\-\_\-point.\-y = faces\-\_\-up\-\_\-left\mbox{[}ii\mbox{]}.point.\-y; face.\-down\-\_\-right\-\_\-point.\-x = faces\-\_\-down\-\_\-right\mbox{[}ii\mbox{]}.point.\-x; face.\-down\-\_\-right\-\_\-point.\-y = faces\-\_\-down\-\_\-right\mbox{[}ii\mbox{]}.point.\-y; response.\-faces.\-push( face ); \}

return response; \} ```

Export the service onrequest callback function ({\ttfamily svc\-Impl})\-:

```js module.\-exports = svc\-Impl ```

Next you will need to create the Web Worker to launch the Web Service\-:

```bash \$ cd $\sim$/rapp\-\_\-platform/rapp-\/platform-\/catkin-\/ws/src/rapp-\/platform/rapp\-\_\-web\-\_\-services/workers \$ touch face\-\_\-detection.\-js ```

Open the {\bfseries face\-\_\-detection.\-js} file with your favorite editor\-:

```bash \$ vim face\-\_\-detection.\-js ```

Import the worker\-Utils module\-:

```js var path = require('path');

var E\-N\-V = require( path.\-join(\-\_\-\-\_\-dirname, '../..', 'env.\-js') );

// Include it even if not used!!! Sets properties to the thread's global scope. var worker\-Utils = require(path.\-join(E\-N\-V.\-P\-A\-T\-H\-S.\-I\-N\-C\-L\-U\-D\-E\-\_\-\-D\-I\-R, 'common', 'worker\-\_\-utils.\-js')); ```

Next, you will have to set the worker name and call to launch all services registred to this Web Worker\-:

```js // Set worker thread name under the global scope. (W\-O\-R\-K\-E\-R.\-name) worker\-Utils.\-set\-Worker\-Name('face\-\_\-detection');

// Launch all services assigned to this worker thread. // Search in workers.\-json config file for assigned web services. worker\-Utils.\-launch\-Svc\-All(); ```

We need to tell the {\itshape run-\/engine} to launch the, newly implemented, Web Worker.

The {\ttfamily workers.\-json} file containes Web Workers entries. It is located under\-:

```bash $\sim$/rapp\-\_\-platform/rapp-\/platform-\/catkin-\/ws/src/rapp-\/platform/rapp\-\_\-web\-\_\-services/config/services ```

Append the following entry in the workers.\-json file\-:

```js \char`\"{}face\-\_\-detection\char`\"{}\-: \{ \char`\"{}launch\char`\"{}\-: true, \char`\"{}path\char`\"{}\-: \char`\"{}workers/face\-\_\-detection.\-js\char`\"{}, \char`\"{}services\char`\"{}\-: \mbox{[} \char`\"{}face\-\_\-detection\char`\"{} \mbox{]} \} ```

Finally append the following {\itshape web-\/service} entry in the services.\-json file (under the same directory)\-:

```js \char`\"{}face\-\_\-detection\char`\"{}\-: \{ \char`\"{}launch\char`\"{}\-: true, \char`\"{}anonymous\char`\"{}\-: false, \char`\"{}name\char`\"{}\-: \char`\"{}face\-\_\-detection\char`\"{}, \char`\"{}url\-\_\-name\char`\"{}\-: \char`\"{}face\-\_\-detection\char`\"{}, \char`\"{}namespace\char`\"{}\-: \char`\"{}\char`\"{}, \char`\"{}authentication\char`\"{}\-: true, \char`\"{}ros\-\_\-connection\char`\"{}\-: true, \char`\"{}timeout\char`\"{}\-: 45000 \} ```

Now the R\-A\-P\-P Platform is ready to receive requests for the newly created face\-\_\-detection service.

```bash \$ cd $\sim$/rapp\-\_\-platform/rapp-\/platform-\/catkin-\/ws/src/rapp-\/platform/rapp\-\_\-web\-\_\-services \$ pm2 start server.\-yaml ```

You will notice the following output from the logs\-:

```bash info\-: \mbox{[}Service Handler\mbox{]} Registered worker service \{\href{http://rapp-platform-local:9001/hop/face_detection}{\tt http\-://rapp-\/platform-\/local\-:9001/hop/face\-\_\-detection}\} under worker thread \{face\-\_\-detection\} info\-: \mbox{[}Service Handler\mbox{]} \{ worker\-: 'face\-\_\-detection', path\-: '/hop/face\-\_\-detection', url\-: '\href{http://rapp-platform-local:9001/hop/face_detection',}{\tt http\-://rapp-\/platform-\/local\-:9001/hop/face\-\_\-detection',} frame\-: \mbox{[}Function\mbox{]} \} ```

\subsection*{Further study}

You can check on already implemented Web Services \href{https://github.com/rapp-project/rapp-platform/tree/master/rapp_web_services/services}{\tt here}.

The R\-A\-P\-P Web\-Service code A\-P\-I is documented \mbox{[}here\mbox{]}().

Documentation of the R\-A\-P\-P Web Services package can be found \href{https://github.com/rapp-project/rapp-platform/tree/master/rapp_web_services}{\tt here}. 